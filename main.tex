\documentclass{beamer}
\usetheme{Boadilla}

%% Code Listings
\usepackage{listings}
\usepackage{xcolor}

\definecolor{codegreen}{rgb}{0,0.6,0}
\definecolor{codegray}{rgb}{0.5,0.5,0.5}
\definecolor{codepurple}{rgb}{0.58,0,0.82}
\definecolor{backcolour}{rgb}{0.95,0.95,0.92}

\lstdefinestyle{mystyle}{
  backgroundcolor=\color{backcolour},
  commentstyle=\color{codegreen},
  keywordstyle=\color{magenta},
  numberstyle=\tiny\color{codegray},
  stringstyle=\color{codepurple},
  basicstyle=\ttfamily\footnotesize,
  breakatwhitespace=false,
  breaklines=true,
  captionpos=b,
  keepspaces=true,
  numbers=left,
  numbersep=5pt,
  showspaces=false,
  showstringspaces=false,
  showtabs=false,
  tabsize=2
}

%% Directory trees
\usepackage{dirtree}

\lstset{style=mystyle}


\title{Introduction to Inheritance and Object Oriented Programming}
\subtitle{Example Instruction Demo}
\author{Tyler Westland}
\date{}
\begin{document}
 
\begin{frame}
\titlepage
\end{frame}

\begin{frame}
\frametitle{Outline}
\tableofcontents
\end{frame}


% Outline
%
%%%%%%%
% Thesis: object oriented programming is a design pattern that links data and functions
%  Understatement: it's the one of the most prolific design pattern, being the basis many programming languages such as Java, Python, and C++
% Circle -- all code
% output of area
% show two circles
% comparison of two circles
% Rectangle -- all code
% show comparison of rectangle and circle
% Shape -- all code
% show shortened comparison
% appendix: folder structure
% appendix: full code of all code

\section{Section 1}
\subsection{sub a}


\begin{frame}
\frametitle{Section 1 - sub a}
Lorem ipsum dolor sit amet, consectetur adipisicing elit, sed do eiusmod tempor incididunt ut labore et dolore magna aliqua.
\end{frame}

\subsection{sub b}

\begin{frame}
\frametitle{Section 1 - sub b}
Blah blah blah blah blah
\end{frame}

\section{Section 2}
\begin{frame}
\frametitle{Section 2}
Stuff
\end{frame}

\appendix
\section{Full Code}
\begin{frame}
\frametitle{Folder Structure}
%\begin{verbatim}
%├─ main.py
%├─ shapes/
%│  ├─ circle.py
%│  ├─ shape.py
%│  ├─ square.py
%\end{verbatim}
\dirtree{%
 .1 code/.
 .2 main.py.
 .2 shapes/.
 .3 shape.py.
 .3 circle.py.
 .3 rectangle.py.
}
\end{frame}

\begin{frame}
\frametitle{main.py}
\lstinputlisting[language=Python, lastline=17]{code/main.py}
\end{frame}

\begin{frame}
\frametitle{main.py -- cont}
\lstinputlisting[language=Python, firstline=18]{code/main.py}
\end{frame}

\begin{frame}
\frametitle{shapes/shape.py}
\lstinputlisting[language=Python]{code/shapes/shape.py}
\end{frame}

\begin{frame}
\frametitle{shapes/circle.py}
\lstinputlisting[language=Python]{code/shapes/circle.py}
\end{frame}

\begin{frame}
\frametitle{shapes/rectangle.py}
\lstinputlisting[language=Python]{code/shapes/rectangle.py}
\end{frame}

\end{document}

