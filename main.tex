\documentclass{beamer}
\usetheme{Boadilla}

%% Code Listings
\usepackage{listings}
\usepackage{xcolor}

\definecolor{codegreen}{rgb}{0,0.6,0}
\definecolor{codegray}{rgb}{0.5,0.5,0.5}
\definecolor{codepurple}{rgb}{0.58,0,0.82}
\definecolor{backcolour}{rgb}{0.95,0.95,0.92}

\lstdefinestyle{mystyle}{
  backgroundcolor=\color{backcolour},
  commentstyle=\color{codegreen},
  keywordstyle=\color{magenta},
  numberstyle=\tiny\color{codegray},
  stringstyle=\color{codepurple},
  basicstyle=\ttfamily\footnotesize,
  breakatwhitespace=false,
  breaklines=true,
  captionpos=b,
  keepspaces=true,
  numbers=left,
  numbersep=5pt,
  showspaces=false,
  showstringspaces=false,
  showtabs=false,
  tabsize=2
}
\lstset{style=mystyle}

%% Directory trees
\usepackage{dirtree}

%% Spacing Commands
\newcommand{\closenewline}{%
  \par\nopagebreak
  \vspace{-\itemsep}
  \vspace{-\parsep}
}
\newcommand{\closeitem}{%
  \closenewline
\item
}



%% Document
\title[Intro to Inheritance \& OOP]{Introduction to Inheritance and Object Oriented Programming}
\subtitle{Example Instruction Demo}
\author{Tyler Westland}
\date{}
\begin{document}
 
\begin{frame}
\titlepage
\end{frame}

\begin{frame}
\frametitle{Outline}
\tableofcontents
\end{frame}


% Outline
%
%%%%%%%
% Thesis: object oriented programming is a design pattern that links data and functions
%  Understatement: it's the one of the most prolific design pattern, being the basis many programming languages such as Java, Python, and C++
% Circle -- all code
% output of area
% show two circles
% comparison of two circles
% Rectangle -- all code
% show comparison of rectangle and circle
% Shape -- all code
% show shortened comparison
% appendix: folder structure
% appendix: full code of all code

\section{High Level}

\begin{frame}
\frametitle{What Problems Do Objects Solve}
\end{frame}

\begin{frame}
\frametitle{What Problems Does Inheritance Solve}
\end{frame}

\section{Examples}
\subsection{The Circle}
  \begin{frame}
  \frametitle{The Circle Class}
  code/shapes/circle.py
  \lstinputlisting[language=Python]{code/shapes/circle.py}
  \end{frame}

  \begin{frame}
  \frametitle{Our First Circle}

  code/main.py
  \lstinputlisting[language=Python, firstline=6, lastline=7]{code/main.py}

  output
  \lstinputlisting[lastline=2]{code/output.txt}

  \begin{block}{Code Snippets}
  main.py and the output will be shown in snippets.
  The full text is in the appendix.
  \end{block}
  \end{frame}

  \begin{frame}
  \frametitle{Two Circles}

  code/main.py
  \lstinputlisting[language=Python, firstline=6, lastline=12]{code/main.py}

  output
  \lstinputlisting[lastline=4]{code/output.txt}
  \end{frame}


\subsection{The Rectangle}
  \begin{frame}
  \frametitle{The Rectangle Class}
  code/shapes/rectangle.py
  \lstinputlisting[language=Python]{code/shapes/rectangle.py}
  \end{frame}

  \begin{frame}
  \frametitle{Comparing Our Shapes}

  code/main.py
  \lstinputlisting[language=Python, firstline=6, lastline=7,numbers=none]{code/main.py}
  \closenewline
  \lstinputlisting[language=Python, firstline=14, lastline=19,numbers=none]{code/main.py}

  output
  \lstinputlisting[lastline=1,numbers=none]{code/output.txt}
  \closenewline
  \lstinputlisting[firstline=4,lastline=5,numbers=none]{code/output.txt}
  \end{frame}

\appendix
\section{Full Code}
  \begin{frame}
  \frametitle{Folder Structure}
  \dirtree{%
   .1 code/.
   .2 main.py.
   .2 shapes/.
   .3 shape.py.
   .3 circle.py.
   .3 rectangle.py.
  }
  \end{frame}

  \begin{frame}
  \frametitle{main.py}
  \lstinputlisting[language=Python, lastline=17]{code/main.py}
  \end{frame}

  \begin{frame}
  \frametitle{main.py -- cont}
  \lstinputlisting[language=Python, firstline=18]{code/main.py}
  \end{frame}

  \begin{frame}
  \frametitle{shapes/shape.py}
  \lstinputlisting[language=Python]{code/shapes/shape.py}
  \end{frame}

  \begin{frame}
  \frametitle{shapes/circle.py}
  \lstinputlisting[language=Python]{code/shapes/circle.py}
  \end{frame}

  \begin{frame}
  \frametitle{shapes/rectangle.py}
  \lstinputlisting[language=Python]{code/shapes/rectangle.py}
  \end{frame}

\subsection{Output}
  \begin{frame}
  \frametitle{output}
  \lstinputlisting[]{code/output.txt}
  \end{frame}

\end{document}

