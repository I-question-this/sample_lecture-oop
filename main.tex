\documentclass{beamer}
%%% Disable pauses
%\documentclass[handout]{beamer}
\usetheme{Darmstadt}

%% Code Listings
\usepackage{listings}
\usepackage{xcolor}

\definecolor{codegreen}{rgb}{0,0.6,0}
\definecolor{codegray}{rgb}{0.5,0.5,0.5}
\definecolor{codepurple}{rgb}{0.58,0,0.82}
\definecolor{backcolour}{rgb}{0.95,0.95,0.92}

\lstdefinestyle{mystyle}{
  backgroundcolor=\color{backcolour},
  commentstyle=\color{codegreen},
  keywordstyle=\color{magenta},
  numberstyle=\tiny\color{codegray},
  stringstyle=\color{codepurple},
  basicstyle=\ttfamily\footnotesize,
  breakatwhitespace=false,
  breaklines=true,
  captionpos=b,
  keepspaces=true,
  numbers=left,
  numbersep=5pt,
  showspaces=false,
  showstringspaces=false,
  showtabs=false,
  tabsize=2
}
\lstset{style=mystyle}

%% Directory trees
\usepackage{dirtree}

%% Spacing Commands
\newcommand{\closenewline}{%
  \par\nopagebreak
  \vspace{-\itemsep}
  \vspace{-\parsep}
}
\newcommand{\closeitem}{%
  \closenewline
\item
}
%% Source
% https://tex.stackexchange.com/questions/51113/horizontal-equivalent-to-raisebox
% Example:
% \begin{document}
% Here is some text. \par
% Here is \shifttext{20pt}{some} text. \par
% Here is \shifttext{-20pt}{some} text. \par
% \end{document}
\newcommand*{\shifttext}[2]{%
  \settowidth{\@tempdima}{#2}%
  \makebox[\@tempdima]{\hspace*{#1}#2}%
}

%%% Other Spacing examples
% some text
% \quad some text
% \qquad some text
% \hspace{0.2\textwidth} some more text
% \hspace{0.333\textwidth} still more text
% \hspace{0.5\textwidth} still more text

%% Fix for Appendix translate error in beamer style
\renewcommand\appendixname{Appendix}
%%% Source
% https://tex.stackexchange.com/questions/192686/hyperref-warning-caused-by-beamer-appendix
%%% Summary
% The problem is the definition of \appendix in beamerbasesection.sty which uses \part{\appendixname} and \appendixname is defined in beamerbasemisc.sty as \translate{Appendix}


%% Document
\title[Intro to Inheritance \& OOP]{Introduction to Inheritance and Object Oriented Programming}
\subtitle{Example Instruction Demo}
\author{Tyler Westland}
\date{}
\begin{document}
 
\begin{frame}
\titlepage
\end{frame}

\begin{frame}
\frametitle{Outline}
\tableofcontents
\end{frame}

\section{High Level}
  \begin{frame}
    \frametitle{What Problems Do Objects Solve}
    \begin{itemize}
      \item<1-> Programs consistent of data\ldots
        \begin{itemize}
          \item Simple: numbers and letters
          \item Complex: measurements associated with timestamps
        \end{itemize}
      \item<2-> and functions
        \begin{itemize} \item Simple: addition and concatenation
          \item Complex: sampling measurements to have timestamps in 30 minute intervals
        \end{itemize}
      \item<3-> Objects combine functions with data
        \begin{itemize}
          \item Simple:\lstinputlisting[language=Python,firstline=1,lastline=2]{examples.py}
          \item Complex: \lstinputlisting[language=Python,firstline=3,lastline=7]{examples.py}
        \end{itemize}
    \end{itemize}
  \end{frame}

  \begin{frame}
    \frametitle{What Problems Does Inheritance Solve}
    \begin{block}{Inheritance}
      The process of creating classes of objects from existing classes
    \end{block}
    \begin{itemize}
      \item<1->Allows for reuse of data structures and functions
        \begin{itemize}
          \item Reused portions can be modified
        \end{itemize}
      \item<2->Describes what to expect from every type of an object
        \begin{itemize}
          \item Every number should allow math operations
          \item Every "Vehicle" in a video game should allow "driving"
        \end{itemize}
    \end{itemize}
    \pause
    Example of a Vehicle being the parent to various drivable objects
    \dirtree{%
     .1 Vehicle.
     .2 Boat.
     .2 Car.
     .2 Plane.
    }
  \end{frame}

\section{Examples}
\subsection{The Generic Shape}
  \begin{frame}
    \frametitle{The Shape Class}
    code/shapes/shape.py
    \lstinputlisting[language=Python, lastline=6]{code/shapes/shape.py}
    \begin{block}{Classes Vs Objects}
      This is a class, it defines what an object is.
      This is distinct from being an instance of an object
    \end{block}
  \end{frame}

\subsection{The Rectangle}
  \begin{frame}
    \frametitle{The Rectangle Class}
    code/shapes/rectangle.py
    \lstinputlisting[language=Python]{code/shapes/rectangle.py}
  \end{frame}

  \begin{frame}
    \frametitle{Our First Rectangle}

    code/main.py
    \lstinputlisting[language=Python, firstline=6, lastline=7]{code/main.py}

    output
    \lstinputlisting[lastline=1]{code/output.txt}
    \begin{block}{Instances of Objects}
      circ is an object as it is an instance of Circle
    \end{block}

    \begin{block}{Code Snippets}
      main.py and the output will be shown in snippets.
      The full text is in the appendix.
    \end{block}
  \end{frame}

  \begin{frame}
    \frametitle{Two Rectangles}

    code/main.py
    \lstinputlisting[language=Python, firstline=6, lastline=12]{code/main.py}

    output
    \lstinputlisting[lastline=3]{code/output.txt}
  \end{frame}


\subsection{The Circle}
  \begin{frame}
    \frametitle{The Circle Class}
    code/shapes/circle.py
    \lstinputlisting[language=Python]{code/shapes/circle.py}
  \end{frame}

  \begin{frame}
    \frametitle{Comparing Our Shapes}

    code/main.py
    \lstinputlisting[language=Python, firstline=6, lastline=7,numbers=none]{code/main.py}
    \closenewline
    \lstinputlisting[language=Python, firstline=14, lastline=19,numbers=none]{code/main.py}

    output
    \lstinputlisting[lastline=1,numbers=none]{code/output.txt}
    \closenewline
    \lstinputlisting[firstline=4,lastline=5,numbers=none]{code/output.txt}
  \end{frame}

\subsection{The Generic Shape: Extended}
  \begin{frame}
    \frametitle{The Extended Shape Class}
    code/shapes/shape.py
    \lstinputlisting[language=Python]{code/shapes/shape.py}
  \end{frame}

  \begin{frame}
    \frametitle{Comparing Our Shapes, With Style}

    code/main.py
    \lstinputlisting[language=Python, firstline=6, lastline=7,numbers=none]{code/main.py}
    \closenewline
    \lstinputlisting[language=Python, firstline=9, lastline=10,numbers=none]{code/main.py}
    \closenewline
    \lstinputlisting[language=Python, firstline=14, lastline=15,numbers=none]{code/main.py}
    \closenewline
    \lstinputlisting[language=Python, firstline=20, lastline=22,numbers=none]{code/main.py}

    output
    \lstinputlisting[lastline=2,numbers=none]{code/output.txt}
    \closenewline
    \lstinputlisting[firstline=4,lastline=4,numbers=none]{code/output.txt}
    \closenewline
    \lstinputlisting[firstline=6,lastline=8,numbers=none]{code/output.txt}
  \end{frame}

\appendix
\section{Full Code}
  \begin{frame}
    \frametitle{Folder Structure}
    \dirtree{%
     .1 code/.
     .2 main.py.
     .2 shapes/.
     .3 shape.py.
     .3 rectangle.py.
     .3 circle.py.
    }
  \end{frame}

  \begin{frame}
    \frametitle{main.py}
    \lstinputlisting[language=Python, lastline=15]{code/main.py}
  \end{frame}

  \begin{frame}
    \frametitle{main.py -- cont}
    \lstinputlisting[language=Python, firstline=16]{code/main.py}
  \end{frame}

  \begin{frame}
    \frametitle{shapes/shape.py}
    \lstinputlisting[language=Python]{code/shapes/shape.py}
  \end{frame}

  \begin{frame}
    \frametitle{shapes/rectangle.py}
    \lstinputlisting[language=Python]{code/shapes/rectangle.py}
  \end{frame}

  \begin{frame}
    \frametitle{shapes/circle.py}
    \lstinputlisting[language=Python]{code/shapes/circle.py}
  \end{frame}

\subsection{Output}
  \begin{frame}
    \frametitle{output}
    \lstinputlisting[]{code/output.txt}
  \end{frame}

\end{document}

